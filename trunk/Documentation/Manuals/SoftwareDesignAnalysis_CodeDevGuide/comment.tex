\chapter{Comments}

TODO \\

\begin{itemize}
 \item HTTP to XML redesing ( HTTP Iterface must be redesigned for generalization )
 \item configuration database must continue support to cfg files but must be go to xml (supports different configuration file types)
 \item streaming to MDSplus
 \item is a good choice to adopt RTTi instead of the GCFilippo support?
 \item group all stuff toghetr with a choerent directory hierical (no levels) with stronger architectures and os support
 \item When XML is done, enable graphical configuration
 \item Switching to exception catching? (try...catch)
 \item code inlining e del codice scritto negli .h ok le performance ma quanto codice ce? quanto codice duplicate?
 \item cleanup the code and file hierical redesign (MARTe, MARTe/MarteSupportLib)
\end{itemize}

Direi che l'architettura � davvero eccezionale.
Or ora sto dedicando del tempo per:
-eventuali altri confronti con altri prodotti commerciali/open source;
-approfondimento di altri linguaggi di programmazione a msg, vedi Smalltalk e ObjC;
in questo modo posso scrivere vantaggi e svantaggi della BaseLib rispetto a questi.

Dal mio punto di vista sarebbe molto interessante poter fare un refeactoring
della BaseLib per poterla esporre anche come prodotto open source.
Vorrei riorganizzarla:
- definendo con pi� chiarezza le interfaccie (molte sono classi astratte), in modo
da SOTTOLINEARE quello che l'aspetto component del sistema;
- far discendere tutte le classi da Object togliendo eventuali loop di inheritance
(che ora come ora non ci sono perch� tu non derivi tutto da Object, ma potrebbero nascere),
in questo modo, assumendo che Object sappia mandare e ricevere messaggi, tutti gli
oggetti possono mandare e ricevere messaggi (come in ObjC) perch� sono figli di Object,
il codice non sarebbe neanche duplicato.
- riorganizzazione delle directory dei sorgenti.

-se gli oggetti avessero anche una completa reflection combinando un db online ed uno offline potresti conoscere tutti i metodi di ogni oggetto e quindi avere una shell tipo VxWorks di debug non solo in kernel